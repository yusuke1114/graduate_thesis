\documentclass[titlepage]{jsreport}

\usepackage[dvipdfmx]{graphicx}
\usepackage[dvipdfmx]{color}
\usepackage{listings}

% jlistingの代わりに以下をプリアンブル(\begin{document}の前)に追加
\renewcommand{\lstlistingname}{ソースコード}
\renewcommand{\lstlistlistingname}{ソースコード目次}
\usepackage{cite}
\usepackage{url}

% ソースコードを挿入するための設定
\lstset{
language={Python},
basicstyle={\ttfamily\small},
backgroundcolor={\color[gray]{.95}},
keywordstyle={\color[rgb]{0.0,0.0,0.8}},
commentstyle={\color[rgb]{0.5,0.5,0.5}},
stringstyle={\color[rgb]{0.0,0.5,0.5}},
frame=single,
numbers=left,
numberstyle={\ttfamily\small},
breaklines=true,
breakindent = 10pt,
tabsize=2,
captionpos=t
}

\title{ゼロインテリジェンストレーダー市場における機械学習トレーダーの影響の解析}
\author{慶應義塾大学理工学部物理情報工学科\\
指導教員 渡辺宙志\\
学籍番号 62219191\\
箕輪勇佑}
\date{2026年3月}

\begin{document}
\pagenumbering{roman}
\maketitle
\setcounter{tocdepth}{2}
\tableofcontents

\chapter{はじめに} \label{chap:introduction}
\pagenumbering{arabic}


\section{研究の背景}

近年、機械学習技術の急速な発展に伴い、金融市場に機械学習が導入されている\cite{Hambly2023,SEZER2020106181}
それを踏まえて機械学習が導入されることによってどのような影響がもたらされるのかという研究が数多く行われている\cite{10.1145/3490354.3494372,Cartea2022,DICKS2024129363,10.1145/3383455.3422570}
例えば、人工知能が導入されるに伴い、市場のトレンド性を弱める一方で 、価格の細かな変動であるボラティリティを増大させるという市場の不安定さが増す報告がされている\cite{10.1145/3490354.3494372}
また、Yaoらは深層強化学習を用いることで、市場の統計的特性を再現している\cite{Yao2024}。しかし、彼らのモデルは高度に複雑化したニューラルネットワークに依存しているため、再現された市場現象の根本的なメカニズムを説明するのは困難となっている。
これに対し、GodeとSunderはエージェントの知能や戦略を極限まで単純化したぜロ・インテリジェンス・トレーダー(ZIT)を提唱した\cite{GSmodel}。
このZITの概念は物理学者たちによって拡張され、Maslovは指値注文市場において、単純なランダム注文の相互作用だけで価格変動のべき乗則の一部が再現可能であることを示した\cite{MASLOV2000571}。
さらに、RabertoらはGenoa Market Modelを提案し、エージェントの予算制約と変動する資産量をモデルに厳密に組み込むことで、ボラティリティ・クラスタリングなどの統計的特徴をより精緻に再現することに成功した\cite{VJUSHIN2001234}
しかし、これらのモデルにおけるエージェントの行動決定は依然としてランダムに近く、市場環境の変化に応じて戦略を意図的に切り替えるような学習の要素は考慮されていない。
つまり、深層強化学習モデルは適応的な振る舞いを再現できる一方でその内部論理がブラックボックスであり、逆にZITやGenoaモデルはメカニズムが明確である一方で状況に合わせて賢く行動を変える能力を欠いている。

\section{研究の目的}

本研究の目的はまず、GodeとSunderのモデルに資産を導入し、学習を持たないZITのみの市場では、資産分布がガウス分布の挙動に留まることを定量的に確認し、これをベースラインとする。
次に、ZITの行動を機械学習させた機械学習プレーヤーを導入した時の市場の影響を調べる。
また、あらかじめ定められた固定的な取引戦略を持つトレーダーをZITの市場に導入した時の影響を比較し、機械学習エージェントが単なる定型的なルールの実行を超え、市場の微細な構造的歪みや需給バランスに適応した戦略を獲得しているかどうかを検証する。

\section{本論文の構成}


\begin{quotation}
    本研究では、ゼロインテリジェンストレーダーで構成される市場に機械学習トレーダーを導入し、その影響について解析を行う。以下に本論文の構成を示す。
    第\ref{chap:introduction}章では、金融市場における機械学習導入の影響と金融市場シミュレーションにおいての問題を示した。
    第\ref{chap:genri}では、本シミュレーションの基盤となるZITモデル、使用したニューラルネットワークについて説明する。
    第\ref{chap:method}章では、構築した市場環境の仕組みと、実装した機械学習トレーダ及びルールベーストレーダーの開発手法について説明する。
    第\ref{chap:results}章では、シミュレーションによって得られた結果について示す。
    第\ref{chap:summary}章では本研究で得られた知見を総括し、結論と今後の展望について述べる。
\end{quotation}

\chapter{原理}\label{chap:genri}

\section{市場構造}

\subsection{注文の種類}
注文の種類は大きく2つに分けられる。1つ目が、指値注文である。指値注文は、「価格」と「数量」を指定して発注する注文である。指定した価格か、それより有利な価格でしか取引せず、即座に取引相手が見つからない場合はオーダーブックに登録され、市場に流動性を供給する役割を果たす。2つ目が、成行注文である。成行注文は、「価格」を指定せず、「数量」のみを指定して発注する注文である。オーダーブックに存在する最良の反対注文と即座に取引を成立させる。確実な取引成立を優先する一方で、価格の決定権を相手にゆだねる形となる。

\subsection{優先順位の原則}
取引所が株の売買を成立させる順序に乗っ取り、価格優先の原則と時間優先の原則によって注文の優先順を定める。

価格優先の原則より、買い注文については値段の高い注文が値段の安い注文に優先し、売り注文については値段の矢数注文が値段の高い注文に優先する。
時間優先の原則より、同じ値段の注文な場合、取引所が受け付けた時間の早い注文が遅い注文に優先する。

\section{ゼロ・インテリジェンス・トレーダー市場}

\subsection{ゼロ・インテリジェンス・トレーダー}
ゼロ・インテリジェンス・トレーダー(以下、ZIT)は、Gode and Sunder (1993) によって提唱されたエージェントモデルである\cite{GSmodel}。ZITは知能や戦略的思考を一切持たない極めて単純なアルゴリズムとして定義される。

ZITは、過去の価格推移(トレンド)、板情報の需給バランス、他者の行動といった市場情報を一切参照しない。また、学習機能や記憶も保持しておらず、その行動は確率的に決定される。
GodeとSunderは、ZITを制約なしモデル(ZI-U)と制約付きモデル(ZI-C)の2種類に分類している。ZI-Uは赤字となる取引も許容し、完全にランダムな価格で注文を出すもでるであり、ZI-Cは利益が負になる注文は出さないという予算制約のみを課したモデルである。本研究では、人間による市場取引の特性に近い挙動を示すことが確認されている ZI-C を採用し、以降単にZITと呼称する。

ZITの意思決定プロセスは、あらかじめ与えられた償還価値(Value)または原価(Cost)に基づく予算制約の範囲内での一様乱数によって決定される。各エージェント $i$ の注文価格決定ルールは2通りある。
買い手の場合、エージェントには、商品の評価額である私的価値 $V_i$ が割り当てられる。買い手は、自身の利益($V_i - P_{bid}$)が確保できる範囲、すなわち $0$ から $V_i$ の間で、一様分布に従ってランダムに注文価格 $P_{bid}$ を決定する。
売り手の場合、エージェントには、商品の調達コストである原価 $C_i$ が割り当てられる。売り手は、自身の利益($P_{ask} - C_i$)が確保できる範囲、すなわち $C_i$ から市場の上限価格 $P_{max}$ の間で、一様分布に従ってランダムに注文価格 $P_{ask}$ を決定する。
このアルゴリズムにより、ZITは安く買って高く売るという戦略的な意図は持たないものの、高く買って安く売るという明白な損失行動は回避される。

\subsection{GSモデル}
GSモデルとは、ZITエージェントを、連続ダブルオークションと呼ばれる市場メカニズムの中で取引させる市場シミュレーションモデルの総称である。
GSモデルにおいて、知能を持たないZITの集団であっても、適切な市場ルールが存在すれば、市場価格は理論的な均衡価格に速やかに収束することを示した。
また、社会的余剰の分配効率は平均97\%に達し、人間のトレーダーによる結果と遜色のない水準を示した。

本研究においてGSモデルを採用する意義は、機械学習を用いた戦略的エージェントの影響を評価するための基準のとしての役割にある。

\section{機械学習}

\subsection{ニューラルネットワーク}
ニューラルネットワークの基本単位は、生物学的なニューロンの働きを簡略化した人工ニューロンである。

各ニューロンは、複数の信号$x_1$,$x_2$,…,$x_n$を受け取り、それぞれの重要度を表す重み$w_i$を乗じて総和をとる。さらに、バイアス$b$を加え、その値を活性化関数$f(\cdot)$に通すことで出力$y$を決定する。この関係式は以下の式で表される。
$$
  y = f\left( \sum_{i=1}^{n} w_i x_i + b \right)
$$

本研究では、これらのニューロンを層状に結合させた多層パーセプトロン(multilayer perceptron, MLP)を採用している。

MLPは、データを受け取る「入力層」、特徴量を抽出する「中間層(隠れ層)」、最終的な予測値を出力する「出力層」から構成される。

\subsection{活性化関数}
ニューラルネットワークにおいて、活性化関数とは、ニューロンが受け取った入力信号を変換し、次のニューロンへの出力を決定する重要な関数である。これにより、ニューラルネットワークは非線形な問題を解決し、複雑なパターンを学習できる。本研究では主に以下の関数が用いられる。\\

\begin{enumerate}
    \item ReLU関数
\end{enumerate}

ReLU(Rectified Linear Unit)とは中間層で一般的に用いられる関数である。入力が0以下の場合は0を出力し、正の場合は入力をそのまま出力する。
$$
f(x) = \max(0, x)
$$
\begin{enumerate}
    \item ソフトマックス関数
\end{enumerate}

ソフトマックス関数とは分類問題における出力層で用いられる関数である。多クラス分類を行う際、出力層の各ニューロンの値を0から1の範囲に正規化し、その総和が 1 になるように変換する。
$$
y_i = \frac{e^{x_i}}{\sum_{j=1}^{K} e^{x_j}}
$$
これにより、ネットワークの出力を確率として解釈することが可能となり、エージェントは最も確率の高い行動を選択することができる。

\subsection{エージェントベースモデル}
エージェントベースモデルは、自律的に意思決定を行うエージェントをコンピュータ上に構築し、それらの相互作用の結果として創発されるマクロな現象を観察する手法である。

\subsection{教師あり学習}
教師あり学習とは、入力データとそれに対応する正解データのペアから、入力と出力の関係を表す関数$f$を学習する手法である。

\subsection{過学習}
モデルが学習データのノイズや細部まで過剰に適合してしまい、未知のデータに対する性能が低下する現象を過学習と呼ぶ。これを防ぐために、データセットを訓練データとテストデータに分割し、モデルの評価を行う手法が用いられる。
図\ref{fig:overfitting}に過学習が起きた時のグラフを示す。
学習初期では、青線も赤線もともに下がっており、モデルがデータの特徴を正しく学習している。緑の点線がロスが最小になる点である。モデルの汎化性能が最も高い状態であり、ここで学習を止めるのがベストとなる。
右側の赤い網掛けでは、青い線は下がり続けているが、赤い線は上昇し始めている。これが過学習であり、モデルが訓練データのノイズまで丸暗記してしまい、新しいデータに対応できなくなっている状態を示している。

\begin{figure}[htbp]
    \centering
    \includegraphics[width=10cm]{fig/overfitting.pdf}
    \caption{過学習}
    \label{fig:overfitting}
\end{figure}

\section{評価指標と統計的性質}

\subsection{ファットテール}
金融市場の価格変動や経済主体の資産分布において、正規分布から想定されるよりも、極端な値が高い頻度で出現する現象をファットテールと呼ぶ。
図\ref{fig:fat}にファットテールを示す。
統計学的に、正規分布の確率密度関数の裾指数関数的に急速に減衰する。
これに対し、現実の複雑系や経済現象に見られる分布の多くは、裾の減衰が緩やかであり、平均値から標準偏差の数倍以上離れた事象が無視できない確率で発生する特性を持つ。


\begin{figure}[htbp]
    \centering
    \includegraphics[width=10cm]{fig/fat_tail.pdf}
    \caption{正規分布とファット・テール分布の比較}
    \label{fig:fat}
\end{figure}

\subsection{相補累積分布関数(CCDF)}
CCDFは累積分布関数$F(x)$を用いて以下のように定義される。
\begin{equation}
    P(X \geq x) = 1 - F(x) = \int_{x}^{\infty} p(t) dt
\end{equation}
ここで、$P(X \geq x)$ は確率変数$X$がある値$x$以上となる確率を表し、$p(t)$は確率密度関数である。
CCDFは確率密度を積分した形式をとるため、データの離散に伴うノイズが平滑化される。
また、CCDFはべき乗則の検証において有効である。対象の分布がパレート分布などのべき分布に従う場合、CCDFを両対数グラフにプロットすると、その軌跡は右下がりの直線を描く。その傾きから、ファットテールの有無を検証することができる。


\chapter{手法} \label{chap:method}

\section{市場環境の構築}
\subsection{市場メカニズム}
本シミュレーションでは、価格形成の基礎的なダイナミクスを抽出するため、Gode and Sunderのモデルをさらに単純化した単一注文保持型の取引メカニズムを採用した。
市場には単一の板が存在し、この場は任意の時点$t$において、3つの状態を取る。1つめが空であり、板に注文が存在しない状態である。2つめが買い気配であり、1人の買い手による買い注文のみが提示されている。
3つめが売り気配であり、売り注文のみが提示されている状態である。本モデルでは、売り注文と買い注文が同時に板に滞留することはなく、直近の最良の未約定注文が1つだけ保持される。
マッチングには、価格優先の原則と時間優先の原則を適用した。

\subsection{場の更新プロセス}
各シミュレーションステップにおいて、ランダムに選出されたエージェントが指値注文を市場に提示することで取引が開始される。この時、提示された注文は板の状態及び、価格と照合され、その結果は「取引成立」「注文の上書き」「注文の棄却」のいずれかの挙動として現れる。

まず、取引成立するケースについて述べる。板に売り注文$P_\mathrm{board}$が滞留している状態で、新たな買い注文$P_mathrm{order}$が$P_b \ge P_mathrm{board}$という条件を満たして提示された場合、即座に売買が成立する。買いと売りが逆のケースも同様に売買が成立する。この際、約定価格は板に待機していた注文の価格が採用される。
約定後は、板に注文が存在しない空の状態へとリセットされる。

次取引に至らない場合の挙動として、注文の上書きと棄却がある。もし、取引が成立せず、新規注文が板の既存注文と比較して価格優先の原則において優位である場合、板の注文は新規注文によって上書きされる。
一方で、新規注文が既存注文よりも価格条件で劣る場合には、その注文は棄却され、板の状態は変化しない。


\section{エージェントモデル}

\subsection{Machine Learning Trader}
Machine Learning Trader (ML Trader)は、市場環境の状態を観測し、取引の成功確率を予測する機械学習モデルを搭載したエージェントである。
まず、価格決定においては戦略的な価格決定は行わず、0~200の間でランダムに価格を選択する。価格を用いて、買い注文を出した場合と売り注文を出した場合のシナリオについて、現在の板情報と組み合わせた入力ベクトルを生成する。
内部に保持する学習済みモデルを用いて、両方のシナリオにおける約定確立を推測する。買い注文と売り注文の約定成功確率を比較し、より高い確率で約定が見込める役割を選択して注文を実行する。

\subsection{Rule-Based Trader}
RB Traderは、現在位の市場の板情報に応じて、確実に有利な配置をとる条件に基づいて行動するエージェントに設定する。
このエージェントは、まず価格をランダムに決定し、その後に市場の板の状態を参照する。板のエージェントが売り手の場合、エージェントの価格が市場の売り気配値以上であれば、即座に取引が可能であるため、買い注文を選択して約定させる。
逆に売り気配値より低い場合は、約定しないため、売り注文を選択して板に新たな指値を提示する。板のエージェントが買い手の場合は逆の行動をする。板が空の場合には、比較対象が存在しないため、ランダムに買い手または売り手を選択して注文を出す。

\section{機械学習モデルの構築}
エージェントが市場環境に適応的な行動を選択するための予測器として、ニューラルネットワークを用いた機械学習モデルを構築する。現在の市場の状態と自信がとろうとする行動を入力し、その注文がどのような結果をもたらすかを確率的に予測する。
順伝番型ニューラルネットワークを採用し、モデル層は入力層、複数の隠れ層および出力層から構成される。入力層は407次元のベクトルを受け取る。隠れ層の活性化関数にはReLUを採用し、出力層にはSoftmax関数を採用した。


\section{シミュレーション}
本実験におけるパラメータ設定を表\ref{tab:sim_params}に示す。1期間あたりのステップ数を4000として、それを100期間シミュレーションした。
シミュレーションは、ZITとMLトレーダーを混ぜて行い、MLトレーダーの割合は10\%~50\%とした。また、ZITとRBトレーダーを混ぜて行い、割合も同様とした。

\begin{table}[htbp]

    \centering
    \caption{シミュレーションのパラメータ設定}
    \label{tab:sim_params}
    \begin{tabular}{lcc}
        \hline
        \textbf{パラメータ} & \textbf{記号} & \textbf{設定値} \\
        \hline
        エージェント総数 & $N$ & 2000 \\
        期間数 & $M$ & 100 \\
        1期間あたりのステップ数 & $T$ & 4000 ($2 \times N$) \\
        初期資産 & $W_0$ & 500 \\
        価格範囲 & $P$ & $[0, 200]$ \\
        \hline
    \end{tabular}
\end{table}

\chapter{結果} \label{chap:results}

\section{ML Traderの特性}

\subsection{取引パフォーマンスと成功率}
機械学習トレーダーの性能、混同行列、正解率など
シミュレーションの結果、MLトレーダーはZITの行動予測について正解率99\%で予測することができた。また、その結果を図\ref{fig:confusion}に混同行列として示す。
縦軸は実際に起きた事象であり、横軸は予測した事象となっている。

\begin{figure}[htbp]
    \centering
    \includegraphics[width=10cm]{fig/confusion_matrix.pdf}
    \caption{混同行列}
    \label{fig:confusion}
\end{figure}


\subsection{ML Traderの資産変動}
図\ref{fig:ccdf_zit}にZITだけでシミュレーションした時の最終資産分布の相補累積を示す。ガウス分布通りの結果であった。
図\ref{fig:cML10}から\ref{fig:cML50}に、MLトレーダーの市場混入率を10\%から50\%まで変化させた際の、最終資産分布の相補累積を示す。
これより、10\%、20\%においてファットテールが出現しており、それ以降ではガウス分布に従っていることが分かった。

\section{ML TraderとRB Traderの比較}
図\ref{fig:cRule10}から\ref{fig:cRule50}に、RBトレーダーの市場混入率を10\%から50\%まで変化させた際の、最終資産分布の相補累積を示す。
RBトレーダーでは、10\%~30\%においてファットテールが出現しており、40\%ではガウス分布通りになっていた。また、50%では再度ファットテールが出現していた。

\subsection{エージェント混入率に対する資産分布パラメータの推移}
MLまたはRBトレーダーの市場全体に対する混入率を10\%から50\%まで変化させた際、最終的な資産分布をガウス分布 $N(\mu, \sigma^2)$ と仮定してフィッティングを行い、そのパラメータ(平均 $\mu$、分散 $\sigma$)の変化を計測した。
ZITのみで市場をシミュレーションした結果、$\mu$は500であり、$\sigma$は720となった。
表\ref{tab:asset_stats_ml}にZITとMLトレーダーの資産分布パラメータ推移を示す。
すべての混入率において、MLトレーダーの資産分布は平均840付近に集中し、ZITの分布を大きく上回る結果となった。
市場の資金がZITからMLトレーダーへと一方的に移転していることをがわかった。また、標準偏差がすべてZITのみでの市場よりも上回っていることがわかった。
表\ref{tab:asset_stats_rule}にZITとRBトレーダーの資産分布パラメータ推移を示す。
市場内の資金がMLトレーダーとは違い、ZITへと一方的に移転していることがわかった。また、標準偏差はすべてZITがRBトレーダーを上回っていた。

\begin{table}[htbp]
    \centering
    \caption{ZITとMLトレーダーの資産分布パラメータ推移(平均 $\mu$, 標準偏差 $\sigma$)}
    \label{tab:asset_stats_ml}
    \begin{tabular}{crrrr}
        \hline
        \textbf{混入率} & \multicolumn{2}{c}{\textbf{ZIT}} & \multicolumn{2}{c}{\textbf{ML Trader}} \\
        (\%) & $\mu$ (平均) & $\sigma$ (標準偏差) & $\mu$ (平均) & $\sigma$ (標準偏差) \\
        \hline
        %% --- Gnuplotの出力をここに貼り付け ---
        10\% & 460.8 & 27.1 & 852.7 & 31.7 \\
        20\% & 419.7 & 27.7 & 821.3 & 33.6 \\
        30\% & 349.1 & 27.9 & 852.0 & 33.1 \\
        40\% & 279.5 & 27.6 & 830.7 & 32.0 \\
        50\% & 160.9 & 28.5 & 839.1 & 30.7 \\
        %% -------------------------------------
        \hline
    \end{tabular}
\end{table}

\begin{table}[htbp]
    \centering
    \caption{ZITとRBトレーダーの資産分布パラメータ推移(平均 $\mu$, 標準偏差 $\sigma$)}
    \label{tab:asset_stats_rule}
    \begin{tabular}{crrrr}
        \hline
        \textbf{混入率} & \multicolumn{2}{c}{\textbf{ZIT}} & \multicolumn{2}{c}{\textbf{RB Trader}} \\
        (\%) & $\mu$ (平均) & $\sigma$ (標準偏差) & $\mu$ (平均) & $\sigma$ (標準偏差) \\
        \hline
        %% --- Gnuplotの出力をここに貼り付け ---
        10\% & 523.4 & 27.0 & 289.6 & 33.1 \\
        20\% & 531.1 & 27.8 & 375.6 & 33.2 \\
        30\% & 536.8 & 28.1 & 414.1 & 33.1 \\
        40\% & 600.6 & 28.8 & 349.1 & 33.0 \\
        50\% & 635.6 & 29.3 & 364.4 & 34.1 \\
        %% -------------------------------------
        \hline
    \end{tabular}
\end{table}

\subsection{トレード履歴}
表\ref{tab:trade_stats_10pct}から\ref{tab:trade_stats_50pct}にZITとMLトレーダーの市場においての行動取引統計を示す。MLトレーダーは高い値段で買い、安い値段で売っているが、売りの回数が買いの回数よりも多いため、利益を得ていることが分かった。
また、表\ref{tab:rule_stats_10pct}から\ref{tab:rule_stats_50pct}にZITとRBトレーダーの市場においての行動取引統計を示す。RBトレーダーは高い値段で買い、安い値段で売っているが、MLトレーダーよりも売りと買いの回数の差が少ないために損をしていることが分かった。
また、図\ref{fig:aML10}から\ref{fig:aML50}にZITとMLトレーダー市場での行動の割合の画像を示す。MLトレーダーはほぼ不成立がないことがわかった。

\begin{table}[htbp]
    \centering
    \caption{ZITとMLトレーダーの行動別取引統計(混入率10\%)}
    \label{tab:trade_stats_10pct}
    \begin{tabular}{lllrrr}
        \hline
        \textbf{Agent Type} & \textbf{Role} & \textbf{Action} & \textbf{Count} & \textbf{Total Price} & \textbf{Mean Price} \\
        \hline
        ZIT & Buyer & Overwrite & 43,281 & 4,039,278 & 93.3 \\
            &       & Executed  & 35,576 & 3,767,048 & 105.9 \\
            \cline{2-6}
            & Seller& Overwrite & 45,552 & 6,069,601 & 133.2 \\
            &       & Executed  & 31,820 & 3,696,517 & 116.2 \\
        \hline
        ML  & Buyer & Overwrite & 4,285 & 604,310 & 141.0 \\
            &       & Executed  & 7,009 & 858,078 & 122.4 \\
            \cline{2-6}
            & Seller& Overwrite & 8,746 & 695,425 & 79.5 \\
            &       & Executed  & 10,765 & 928,609 & 86.3 \\
        \hline
    \end{tabular}
\end{table}

\begin{table}[htbp]
    \centering
    \caption{ZITとMLトレーダーの行動別取引統計(混入率20\%)}
    \label{tab:trade_stats_20pct}
    \begin{tabular}{lllrrr}
        \hline
        \textbf{Agent Type} & \textbf{Role} & \textbf{Action} & \textbf{Count} & \textbf{Total Price} & \textbf{Mean Price} \\
        \hline
        ZIT & Buyer & Overwrite & 43,394 & 3,900,929 & 89.9 \\
            &       & Executed  & 35,991 & 3,578,930 & 99.4 \\
            \cline{2-6}
            & Seller& Overwrite & 45,539 & 6,155,463 & 135.2 \\
            &       & Executed  & 29,314 & 3,450,420 & 117.7 \\
        \hline
        ML  & Buyer & Overwrite & 8,282 & 1,153,394 & 139.3 \\
            &       & Executed  & 14,789 & 1,684,748 & 113.9 \\
            \cline{2-6}
            & Seller& Overwrite & 18,857 & 1,506,930 & 79.9\\
            &       & Executed  & 21,466 & 1,813,258 & 84.5\\
        \hline
    \end{tabular}
\end{table}

\begin{table}[htbp]
    \centering
    \caption{ZITとMLトレーダーの行動別取引統計(混入率30\%)}
    \label{tab:trade_stats_30pct}
    \begin{tabular}{lllrrr}
        \hline
        \textbf{Agent Type} & \textbf{Role} & \textbf{Action} & \textbf{Count} & \textbf{Total Price} & \textbf{Mean Price} \\
        \hline
        ZIT & Buyer & Overwrite & 42,370 & 3,656,729 & 86.3 \\
            &       & Executed  & 35,748 & 3,339,293 & 93.4 \\
            \cline{2-6}
            & Seller& Overwrite & 45,442 & 6,210,620 & 136.7 \\
            &       & Executed  & 26,583 & 3,128,095 & 117.7 \\
        \hline
        ML  & Buyer & Overwrite & 11,745 & 1,614,503 & 137.5 \\
            &       & Executed  & 23,160 & 2,469,765 & 106.6 \\
            \cline{2-6}
            & Seller& Overwrite & 30,672 & 2,485,957 & 81.0 \\
            &       & Executed  & 32,325 & 2,680,963 & 82.9\\
        \hline
    \end{tabular}
\end{table}

\begin{table}[htbp]
    \centering
    \caption{ZITとMLトレーダーの行動別取引統計(混入率40\%)}
    \label{tab:trade_stats_40pct}
    \begin{tabular}{lllrrr}
        \hline
        \textbf{Agent Type} & \textbf{Role} & \textbf{Action} & \textbf{Count} & \textbf{Total Price} & \textbf{Mean Price} \\
        \hline
        ZIT & Buyer & Overwrite & 40,158 & 3,353,007 & 83.5 \\
            &       & Executed  & 34,932 & 3,090,276 & 88.5 \\
            \cline{2-6}
            & Seller& Overwrite & 44,029 & 6,096,027 & 138.5 \\
            &       & Executed  & 23,824 & 2,825,728 & 118.6 \\
        \hline
        ML  & Buyer & Overwrite & 14,360 & 1,978,542 & 137.8 \\
            &       & Executed  & 32,220 & 3,237,738 & 100.5 \\
            \cline{2-6}
            & Seller& Overwrite & 44,293 & 3,600,807 & 81.3 \\
            &       & Executed  & 43,328 & 3,502,286 & 80.8 \\
        \hline
    \end{tabular}
\end{table}

\begin{table}[htbp]
    \centering
    \caption{ZITとMLトレーダーの行動別取引統計(混入率50\%)}
    \label{tab:trade_stats_50pct}
    \begin{tabular}{lllrrr}
        \hline
        \textbf{Agent Type} & \textbf{Role} & \textbf{Action} & \textbf{Count} & \textbf{Total Price} & \textbf{Mean Price} \\
        \hline
        ZIT & Buyer & Overwrite & 37,276 & 3,015,605 & 80.9 \\
            &       & Executed  & 33,318 & 2,777,456 & 83.4 \\
            \cline{2-6}
            & Seller& Overwrite & 42,151 & 5,875,114 & 139.4 \\
            &       & Executed  & 20,723 & 2,438,384 & 117.7 \\
        \hline
        ML  & Buyer & Overwrite & 16,405 & 2,235,514 & 136.3 \\
            &       & Executed  & 41,816 & 3,925,617 & 93.9 \\
            \cline{2-6}
            & Seller& Overwrite & 59,687 & 4,885,464 & 81.9 \\
            &       & Executed  & 54,411 & 4,264,689 & 78.4 \\
        \hline
    \end{tabular}
\end{table}

\begin{table}[htbp]
    \centering
    \caption{ZITとRBトレーダーの行動別取引統計(混入率10\%)}
    \label{tab:rule_stats_10pct}
    \begin{tabular}{lllrrr}
        \hline
        \textbf{Type} & \textbf{Role} & \textbf{Action} & \textbf{Count} & \textbf{Sum} & \textbf{Mean} \\
        \hline
        ZIT & 買い & Overwrite & 44,646 & 4,193,447 & 93.9 \\
            &      & Executed  & 34,892 & 3,766,356 & 107.9 \\
            \cline{2-6}
            & 売り & Overwrite & 44,452 & 5,954,731 & 134.0 \\
            &      & Executed  & 32,128 & 3,786,768 & 117.9 \\
        \hline
        RB  & 買い & Overwrite & 5,789 & 765,179 & 132.2 \\
            &      & Executed  & 7,525 & 950,777 & 126.3 \\
            \cline{2-6}
            & 売り & Overwrite & 7,161 & 541,611 & 75.6 \\
            &      & Executed  & 10,289 & 930,365 & 90.4 \\
        \hline
    \end{tabular}
\end{table}

\begin{table}[htbp]
    \centering
    \caption{ZITとRBトレーダーの行動別取引統計(混入率20\%)}
    \label{tab:rule_stats_20pct}
    \begin{tabular}{lllrrr}
        \hline
        \textbf{Type} & \textbf{Role} & \textbf{Action} & \textbf{Count} & \textbf{Sum} & \textbf{Mean} \\
        \hline
        ZIT & 買い & Overwrite & 45,743 & 4,164,043 & 91.0 \\
            &      & Executed  & 34,937 & 3,590,922 & 102.8 \\
            \cline{2-6}
            & 売り & Overwrite & 44,010 & 5,992,250 & 136.2 \\
            &      & Executed  & 30,166 & 3,629,491 & 120.3 \\
        \hline
        RB  & 買い & Overwrite & 12,514 & 1,618,941 & 129.4 \\
            &      & Executed  & 15,770 & 1,917,296 & 121.6 \\
            \cline{2-6}
            & 売り & Overwrite & 14,908 & 1,147,464 & 77.0 \\
            &      & Executed  & 20,541 & 1,878,727 & 91.5 \\
        \hline
    \end{tabular}
\end{table}

\begin{table}[htbp]
    \centering
    \caption{ZITとRuleトレーダーの行動別取引統計(混入率30\%)}
    \label{tab:rule_stats_30pct}
    \begin{tabular}{lllrrr}
        \hline
        \textbf{Type} & \textbf{Role} & \textbf{Action} & \textbf{Count} & \textbf{Sum} & \textbf{Mean} \\
        \hline
        ZIT & 買い & Overwrite & 44,567 & 3,948,665 & 88.6 \\
            &      & Executed  & 34,120 & 3,354,155 & 98.3 \\
            \cline{2-6}
            & 売り & Overwrite & 43,384 & 6,003,130 & 138.4 \\
            &      & Executed  & 27,992 & 3,439,141 & 122.9 \\
        \hline
        RB  & 買い & Overwrite & 19,822 & 2,522,978 & 127.3 \\
            &      & Executed  & 24,653 & 2,911,869 & 118.1 \\
            \cline{2-6}
            & 売り & Overwrite & 22,849 & 1,766,884 & 77.3 \\
            &      & Executed  & 30,781 & 2,826,883 & 91.8 \\
        \hline
    \end{tabular}
\end{table}

\begin{table}[htbp]
    \centering
    \caption{ZITとRBトレーダーの行動別取引統計(混入率40\%)}
    \label{tab:rule_stats_40pct}
    \begin{tabular}{lllrrr}
        \hline
        \textbf{Type} & \textbf{Role} & \textbf{Action} & \textbf{Count} & \textbf{Sum} & \textbf{Mean} \\
        \hline
        ZIT & 買い & Overwrite & 42,956 & 3,688,280 & 85.9 \\
            &      & Executed  & 32,337 & 3,064,667 & 94.8 \\
            \cline{2-6}
            & 売り & Overwrite & 41,601 & 5,828,530 & 140.1 \\
            &      & Executed  & 25,563 & 3,183,123 & 124.5 \\
        \hline
        RB  & 買い & Overwrite & 27,155 & 3,411,213 & 125.6 \\
            &      & Executed  & 34,249 & 3,924,290 & 114.6 \\
            \cline{2-6}
            & 売り & Overwrite & 31,299 & 2,465,202 & 78.8 \\
            &      & Executed  & 41,023 & 3,805,834 & 92.8 \\
        \hline
    \end{tabular}
\end{table}

\begin{table}[htbp]
    \centering
    \caption{ZITとRBトレーダーの行動別取引統計(混入率50\%)}
    \label{tab:rule_stats_50pct}
    \begin{tabular}{lllrrr}
        \hline
        \textbf{Type} & \textbf{Role} & \textbf{Action} & \textbf{Count} & \textbf{Sum} & \textbf{Mean} \\
        \hline
        ZIT & 買い & Overwrite & 41,066 & 3,441,163 & 83.8 \\
            &      & Executed  & 30,237 & 2,780,731 & 92.0 \\
            \cline{2-6}
            & 売り & Overwrite & 38,873 & 5,513,323 & 141.8 \\
            &      & Executed  & 20,723 & 2,925,478 & 126.4 \\
        \hline
        RB  & 買い & Overwrite & 36,052 & 4,478,623 & 124.2 \\
            &      & Executed  & 44,567 & 5,010,930 & 112.4 \\
            \cline{2-6}
            & 売り & Overwrite & 39,851 & 3,179,292 & 79.8 \\
            &      & Executed  & 51,651 & 4,866,183 & 94.2 \\
        \hline
    \end{tabular}
\end{table}


\begin{figure}[htbp]
    \centering
    \includegraphics[width=10cm]{fig/ccdf_zit.pdf}
    \caption{ZIT}
    \label{fig:ccdf_zit}
\end{figure}

\begin{figure}[htbp]
    \centering
    \includegraphics[width=10cm]{fig/ccdf_10.pdf}
    \caption{MLトレーダー(10\%)}
    \label{fig:cML10}
\end{figure}

\begin{figure}[htbp]
    \centering
    \includegraphics[width=10cm]{fig/ccdf_20.pdf}
    \caption{MLトレーダー(20\%)}
    \label{fig:cML20}
\end{figure}

\begin{figure}[htbp]
    \centering
    \includegraphics[width=10cm]{fig/ccdf_30.pdf}
    \caption{MLトレーダー(30\%)}
    \label{fig:cML30}
\end{figure}

\begin{figure}[htbp]
    \centering
    \includegraphics[width=10cm]{fig/ccdf_40.pdf}
    \caption{MLトレーダー(40\%)}
    \label{fig:cML40}
\end{figure}

\begin{figure}[htbp]
    \centering
    \includegraphics[width=10cm]{fig/ccdf_50.pdf}
    \caption{MLトレーダー(50\%)}
    \label{fig:cML50}
\end{figure}

\begin{figure}[htbp]
    \centering
    \includegraphics[width=10cm]{fig/ccdf_R_10.pdf}
    \caption{RBトレーダー(10\%)}
    \label{fig:cRule10}
\end{figure}

\begin{figure}[htbp]
    \centering
    \includegraphics[width=10cm]{fig/ccdf_R_20.pdf}
    \caption{RBトレーダー(20\%)}
    \label{fig:cRule20}
\end{figure}

\begin{figure}[htbp]
    \centering
    \includegraphics[width=10cm]{fig/ccdf_R_30.pdf}
    \caption{RBトレーダー(30\%)}
    \label{fig:cRule30}
\end{figure}

\begin{figure}[htbp]
    \centering
    \includegraphics[width=10cm]{fig/ccdf_R_40.pdf}
    \caption{RBトレーダー(40\%)}
    \label{fig:cRule40}
\end{figure}

\begin{figure}[htbp]
    \centering
    \includegraphics[width=10cm]{fig/ccdf_R_50.pdf}
    \caption{RBトレーダー(50\%)}
    \label{fig:cRule50}
\end{figure}

\begin{figure}[htbp]
    \centering
    \includegraphics[width=10cm]{fig/action_ML10.pdf}
    \caption{MLトレーダー(10\%)}
    \label{fig:aML10}
\end{figure}

\begin{figure}[htbp]
    \centering
    \includegraphics[width=10cm]{fig/action_ML20.pdf}
    \caption{MLトレーダー(20\%)}
    \label{fig:aML20}
\end{figure}

\begin{figure}[htbp]
    \centering
    \includegraphics[width=10cm]{fig/action_ML30.pdf}
    \caption{MLトレーダー(30\%)}
    \label{fig:aML30}
\end{figure}

\begin{figure}[htbp]
    \centering
    \includegraphics[width=10cm]{fig/action_ML40.pdf}
    \caption{MLトレーダー(40\%)}
    \label{fig:aML40}
\end{figure}

\begin{figure}[htbp]
    \centering
    \includegraphics[width=10cm]{fig/action_ML50.pdf}
    \caption{MLトレーダー(50\%)}
    \label{fig:aML50}
\end{figure}

\chapter{考察および結論} \label{chap:summary}
\section{ファットテールの出現}
本研究では、ランダムに取引を行うZIトレーダーで構成された市場に、取引確率の最大化を学習した機械学習トレーダーが参入した際の影響を検証した。先行研究では、アルゴリズム取引の導入は主に市場のボラティリティの増大に寄与すると報告されており、本研究でも同様の不安定化を予想していた。しかし、シミュレーションの結果、ボラティリティの変化よりも、市場参加者の最終資産分布においてファットテールが現れ、資産の統計量に大きな変化が現れることが分かった。
この現象の発生メカニズムを解明するため、エージェント属性ごとの資産分布を解析したところ、ZIトレーダーとMLトレーダーは、それぞれ平均と分散が異なるガウス分布を形成していることが判明した。すなわち、市場全体で観測されたファットテールは、単一のべき乗則によるものではなく、特性の異なる複数のガウス分布の混合によって形成された現象であることが分かった。
現実の資産分布におけるファットテールは富の偏りによって起きると考えられているため、本シミュレーションのファットテールとは出現メカニズムが異なると考えられる。

\section{MLトレーダーとRBトレーダーの比較}
本研究では、MLトレーダーが獲得した戦略が、相手にとって有利な価格を提示するという単純なルールに基づいていることから、この行動を模倣したRBトレーダーを実装し、比較実験を行った。
ここで、シミュレーションの基盤となるZITエージェントのパラメータ生成プロセスにおいて、取引の成立を促進するために評価額(Value)が原価(Cost)を下回る場合、Valueを再設定するという補正処理が行われている。
この仕様により、市場全体として買い手の予算(Value)の平均が中央値である100よりも高めに設定される傾向が生じた。その結果、板上には高値の買い注文が滞留しやすく、安値の売り注文は約定し枯渇しやすいという買い需要過多・売り供給不足の構造的バイアスが常態化していた。
この市場環境下において、行動確率の最大化を目指すMLトレーダーと確実に行動するRBトレーダーはそれぞれ異なる適応を見せた。RBトレーダーは市場に買い注文が多くなるため、確率的に売り注文が約定しやすくなる。その結果、RBトレーダーの行動は自然と売りに偏った。
しかし、板が空の時においては買い手になるのか売り手になるのかはランダムであり、高く買って安く売るコストを相殺するまでには至らず、純資産を減少させたと考えられる。一方、MLトレーダーは学習過程において買うことを避け、売りを多くするという非対称戦略を能動的に選択した。MLトレーダーは板が空の時でも売りに特化することは適用される。
この板が空の時の行動の差によってMLトレーダーは利益を得ることができたと考えられる。

\section{結論と今後の展望}
本研究では、エージェントベースモデルを用いて、機械学習トレーダーがZIT市場に与える影響を定量的に評価した。市場のボラティリティについては大きな変化は見られなかった一方で、資産分布においてはファットテールの出現が確認された。
機械学習エージェントは市場の構造的な歪み、具体的にはZITのパラメータ生成ルールに起因する買い需要の滞留を検知し、買いを抑えつつ売りを増やすという戦略を獲得したことが明らかとなった。
この戦略の結果、市場参加者は売り特化により現金を回収する機械エージェントとそれを受け止めるZITという取引行動や資産推移の統計的性質が異なる二つのクラスへと分極化し、この平均や分散の異なる分布の混合が市場全体としてみた際にファットテールとして観測されたと推測できる。

今後は強化学習によるエージェント導入に加え、注文のキャンセル機能や複数単位取引の実装をすることにより、シミュレーション環境を現実へと近づけていく。そのうえで、より一般的な市場条件下における富の偏在メカニズムの解明や説明可能なAI技術を用いた戦略の可視化に取り組み、機械学習が金融市場の安定性や効率性に与える影響を多角的に評価することを目指す。
\chapter*{謝辞}

初めに指導教員である渡辺宙志先生に、心より感謝いたします。金融という物理情報工学科とは遠い分野にもかかわらず、テーマの考案から卒業論文の執筆に至るまでの助言や、プログラミングやソフトウェアについての指導など、私の研究に大いに役立ちました。
また、また,研究者としてあるべき姿の手本となってくださった研究室の先輩方と,共に切磋琢磨した研究室の同期にも感謝いたします。

最後に、大学卒業に至るまで親身にサポートしてくれた両親に心から感謝を申し上げます。本研究に取り組むことができたのは、大学進学や日常生活において自由な選択肢を与えてくれた両親のおかげです。本当にありがとうございます。
\appendix

\chapter{ソースコード}

%\lstinputlisting[caption = ML.py, label = ML]{src/ML.py}
%\lstinputlisting[caption = Rule.py, label = Rule]{src/Rule.py}
%\lstinputlisting[caption = ZIT.py, label = ZIT]{src/ZIT.py}
%\lstinputlisting[caption = zit_simulation.py, label = ZIT_s]{src/zit_simulation.py}
%\lstinputlisting[caption = zit_train.py, label = ZIT_t]{src/zit_train.py}

\bibliographystyle{junsrt}

\bibliography{reference}

\end{document}